\begin{verbatim}
Lieber Eugen,
hier unsere Notizen.
Regelmäßiges Treffen: Mittwoch 16 Uhr.

Ziel: 70 Seiten. Unter 50 ist zu wenig...

1. 31. Mai Einleitung, Motivation & Ziele, Strukturskizze der Arbeit
(grob). => 4 Seiten
Zusätzlich Skizze für DesignKapitel mit Stichpunkten (d.h. weiter
ausführen von Kapitel 3., evtl. erste Bilder)

2. 31.05-05.06
Related Work und Hintergrund (7 Seiten)
* Machine Learning (Features, Labels), Classification, Clustering,
K-Cross-Validation
* Verhalten von I/O Systemen (Caches, Wie läuft I/O Kommunikation ab)
* SIOX => Architektur aus Papern bzw. dein Wissen klauen.
Bisher kein Vollständiger Wissenszyklus (Messen => Lernen =>
Optimieren => Optimieren ...)

06.06-13.06
3. Design => ca. 15 Seiten
- Optimierungs/Wissenszyklus Abstrakt
- Leistungs-Vorhersage (Predictor)
- Reduktion von Werten zu Klassen => Preprocessing
- Ursachen-Klassifikator (Cached vs. Uncached mit Labels die von
Benutzern zu definieren sind). Beispielgraph einfach von Daten
übernehmen.
- Extraction von Wissen aus Decision Trees => verweis nach hinten.
- Einsatz von Decision Tree (parameterisierung) ??? Sonst Implementierung.
- Beispielhafte Ausgabe eines Programlaufs
=> Bei Terminierung des Programs wollen wir ja sehen, dass der die
Gründe und Bewertung vornimmt. Wieviel I/O In Klasse 1, 2....
(Entsprechend der Namen).

4. Implementation (kurz), ca 5 Seiten
- Ich habe rumprobiert mit Shark, OpenCV etc. => Vor & Nachteile
einzelner Bibliotheken => Warum hast du dich für Shogun entschieden.
Eigene Studie.
- Wissensextraktionsalgorithmus (Der nicht ging)

14.06  - 21.06:
5. Evaluation => ca. 15 Seiten
- Trainingsdaten (welche woher kommen die, wie sehen die Systeme aus...)
- Genauigkeit der Lösung (K-Cross-validation)
- Betrachtung der Regeln (Bäume malen)
- Verhalten bei weniger Trainingsdaten (Inverse Cross-Valdiation).
- Leistungsverhalten des Machine LEarnings (Wieviel Performance kostet das)

6. Zusammenfassung 1 Seite (hierbei nochmal Einleitung, Hintergrund,
Related Work schärfen auf Ziele, Struktur in Einleitung anpassen)
Ausblick => Nachfolgende Arbeiten (Was ist noch zu tun).


22-29.06 Korrekturen vornehmen und evtl. Experimente noch ausführen.
Drucken 29. Juni
Abgabe 30. Juni

Kapitel immer an Julian schicken, bzw. in GIT Bereitstellen, FAST
fertige Kapitel (80%).

Bitte Kapitelüberleitungen (kurz was habe ich im Kapitel gelernt, was
kommt in nächsten Kapitel).
In Kursiv schreiben.
Zu beginn eines Kapitels, die Struktur des Kapitels erläutern
(Kursiv), d.h. der Rote Faden wird immer klar, aber Inhalt hat dort
NICHTS zu suchen.

Grüße,
Julian

-- http://wr.informatik.uni-hamburg.de/people/julian_kunkel 
\end{verbatim}
