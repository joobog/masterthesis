%2. 31.05-05.06
%Related Work und Hintergrund (7 Seiten)
%* SIOX => Architektur aus Papern bzw. dein Wissen klauen.
%Bisher kein Vollständiger Wissenszyklus (Messen => Lernen =>
%Optimieren => Optimieren ...)


\newpage

\section{Verwandte Arbeiten und Hintergrund}
\textit{\todo{Was kommt im Kapitel kurz die (Unter-)Gliederung darstellen}, für den Leser ! hier mal ein Beispiel:}

%\textit{
%In diesem Kapitel werden zunächst relevante Forschungsarbeiten aus dem Bereich der Leistungsanalyse und Optimierung von Ein/Ausgabe vorgestellt.
%In Kapitel... wird ... ...
%Vampir in Section...
%SIOX, auf dem diese Arbeit basiert, wird in Kapitel X beleuchtet.
%Ansätze welche Maschinelles Lernen im HPC bereits einsetzen sind in...

%Des weiteren wird eine Einführung in die Thematik des Maschinellen Lernens in X gegeben und in Y relevante Faktoren für E/A-Leistung.
%}


\todo{Der folgende Text hier gehört gemerged mit dem Intro. Hier gehört das NICHT her...}
Die Leistungsanalyse und -optimierung von Anwendungen sind seit dem Beginn der elektronischen Datenverarbeitung intensive Forschungsobjekte. 
Seit die Engpässe \todo{Was? Welche?} so groß geworden sind, schenkt man den mehr Beachtung. 
Ohne geeignete Werkzeuge ist heute kaum möglich eine Leistungsanalyse durchzuführen. 
Programme können schon auf typischen Desktoprechnern tausende von E/A-Operation pro Sekunde durchführen und auf den heutigen HPC-System können es sogar hunderttausende oder Millionen sein. 
Die zu analysierende Datenmengen werden dabei nicht selten hunderte von Gigabytes gross. 
Um das komplexe Zusammenspiel der E/A-Zugriffe dennoch zu verstehen und um die Ursache dafür zu identifizieren werden bringt man den Tools immer cleverere Analyse-Techniken bei.


\subsection{Darshan}
Darshan ist ein Charakterisierungstool für HPC E/A.
Es wurde entworfen um ein genaues Bild vom Anwendungsverhalten und Eigenschaften, wie Zugriffsmuster in einer Datei, mit minimalen  Overhead aufzunehmen.
\todo{Referenz, schau mal mit google scholar, da geht das in Minuten}

Darshan kann benutzt werden um das E/A-Verhalten von HPC-Anwendungen zu untersuchen und zu optimieren. 
Sein leichter Design macht ihn auch optimal für die durchgehende Lastcharakterisierung in großen Systemen während des produktiven Betriebs.

\todo{Hier sollte noch mehr stehen: was Darshan für Reports ausgeben kann, wie funktioniert es, zur Laufzeit erfassung der API-Aufrufe von bspw. POSIX und MPI, die Dateinamen mit größtem Overhead werden ausgegeben, es erfasst größe der E/A und Random vs. Sequential ... Siehe Paper!}

\subsection{Vampir}
%%% Vampir %%% 
% Kurze Beschreibung von Vampir
% graphische Software
% unterstuetzt MPI, OpenMP, multithreaded Applications
Vampir \todo{Referenz} ist ein grafisches Werkzeug zur Leistungsanalyse von parallelen Systemen. 
Es unterstützt die Offline-Analyse von parallelen (MPI, OpenMP, Multithreading) und hardwarebeschleunigten (CUDA und OpenCL) Anwendungen. 
Seine Analyseengine erlaubt eine skalierbare und effiziente Verarbeitung von großen Mengen von Leistungsdaten.

% Funktionsweise
% erstellt Traces
Vampir nutzt die Infrastruktur von Score-P \todo{Referenz} zum Instrumentieren von Anwendungen. 
Score-P speichert die Aktivitäten in einer Datei ab, die von Vampir analysiert und in verschieden Sichten umgewandelt werden kann, z.B. können die Aktivitäten auf einer Zeitachse dargestellt werden oder zu einer der vielen Statistiken komprimiert werden. 
Einige Sichten verfügen über aufwendige Filter- und Zoom-Funktionen, mit den man sich leicht einen Überblick verschaffen und die Details betrachten kann.  

% Vor- und Nachteile
Eine effektive Nutzung von Vampir erfordert ein tiefes Hintergrundwissen in der parallelen Programmierung. 
Das Program ermöglicht die Aufzeichung und Auswertung von POSIX-E/A Zugriffen, aber wenig Informationen über deren Zustandekommen oder zur Bewertung der E/A. 
Der Einsatzbereich von Vampir wird durch die fehlende Unterstützung der Online-Analyse etwas eingeschränkt.  

% Zweck:
% Senken von Kommunikationskosten
% Balancierung von Kommunikation zwischen CPU und Cache oder Arbeitsspeicher
% Engpaesserkennug auf Kommunikationsstrecken
%Vampir eignet sich wenn es darum geht die Kommunikationskosten zu senken, die Kommunikation wischen CPU und Cache bzw. Arbeitsspeicher auszubalancieren oder Engpässe auf Kommunikationsstrecken zu erkennen.


\subsection{SIOX-Framework}

% Allgemeine Beschreibung
% Unterstützung weiterer Interfaces
% Activitaetgenerator
% Monitoring Plugins
SIOX \cite{siox_arch} ist ein unter GPL veröffentlichte Open-Source-Framework, der seit dem Jahr 2011 im Rahmen eines Forschungsprojekts des BMBF an den Universitäten Hamburg, Dresden und Stuttgart entwickelt wurde.
Es unterstützt neben der ereignisbasierte Erfassung und Analyse von E/A-Zugriffe auf Desktop- und HPC-Systemen auch die Möglichkeit die Ein/Ausgabe zu optimieren. 
Das Framework enthält Werkzeuge für die Erstellung der Instrumentation von Schnittstellen, zur Auswertung von Spurdaten und diverse Analyse-Plugins. 

% Funktionsweise
% Aktivitätene und Sequenzen
Ähnlich wie Vampir, arbeitet SIOX mit Aktivitäten und Aktivitätssequenzen, die beim Ausführen der instrumentierten Anwendungen erfasst werden oder künstlich mit einem Plugin erzeugt werden können. 
Diese Aktivitäten sind Datenstrukturen, die relevanten Daten und Messungen über die E/A-Operation beinhalten. 
Sie können zur späteren Analyse in einer Datei gespeichert oder in Echtzeit verarbeitet werden.

% Plugins
Im Kern stellt SIOX eine Plugin-Infrastruktur bereit und erhält seine entgültige Funktionalität erst durch die Anbindung von Plugins und Modulen. 
Das macht SIOX leicht erweiterbar und fast unbegrenzt an die eigene Bedürfnisse anpassbar. 
So kann man z.B. eigene Module zur Instrumentierung von Schnittstellen anbinden oder die Aktivitäten künstlich mit einem eigenen Aktivitätsgenerator-Plugin erzeugen und die erzeugten Aktivitäten mit eigenen Analyse-Plugins verarbeiten.

% GUI und Ausgabe
Ebenfalls kann SIOX bei der Terminierung einer Anwendung relevante statistische Information zur E/A ausgeben in vergleichbarer Form zu Darshan.
%In SIOX ist es möglich Informationen im Web-Interface anzuzeigen. 
%Diese Funktion ist momentan nur auf die Auflistung von (Cached vs. Uncached mit Labels die von\\Aktivitäten beschränkt, kann aber leicht erweitert werden. 
Komplexere Ausgaben und Auswertungen können durch die Plugins vorgenommen werden und neue Informationen bei der Programmterminierung ausgegeben.
% DAS NUTZT DU JA!

% Vorteile und Nachteile
SIOX unterstützt sowohl Online- als auch Offline-Analyse. 
Zusätzlich erlaubt die Anbindung eines Aktivitätsgenerator-Plugins auch eine hypothetische Situation zu simulieren. 
Das Framework befindet sich im Entwicklungsstadium und hat keine feste Schnittstelle und verfügt bislang über wenige Plugins.


%\subsection{Kombination vom analytischen und maschinellen Lernen}
%\todo{Willst du das wirklich zitieren, ich würde das weg lassen}

%In \cite{Didona:2015:EPP:2668930.2688047} versuchen die Autoren eine Verbesserung der Leistungsvorhersage mit einer Kombination aus Machine-Learning und Analytical Modeling zu erreichen und Vorteile aus beiden Modellen zu nutzen. 
%Aus dem analytischen Modell kann leicht herauslesen, wie die Entscheidung zustande kommt, aber das Modell liefert keine so gute Vorhersagen wie Machine-Learning.
%Das Machine-Learning liefert zwar wesentlich bessere Ergebnisse, ist aber das Zustandekommen der Entscheidung ist sehr schwer nachvollziehbar. \todo{Zusammenfassen}

\bigskip

\textit{\todo{Kurze Zusammenfassung und Überleitung, ein paar Zeilen}}
