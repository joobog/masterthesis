%2. 31.05-05.06
%Related Work und Hintergrund (7 Seiten)
%* SIOX => Architektur aus Papern bzw. dein Wissen klauen.
%Bisher kein Vollständiger Wissenszyklus (Messen => Lernen =>
%Optimieren => Optimieren ...)
\section{Verwandte Arbeiten und Hintergrund}
Die Leistungsanalyse und -optimierung sind seit dem Beginn der elektronischen Datenverarbeitung intensive Forschungsobjekte. 
Insbesondere seit die Engpässe so groß geworden sind, schenkt man den mehr Beachtung. 
Ohne geeignete Werkzeuge ist heute kaum möglich eine Leistungsanalyse durchzuführen. 
Programme können schon auf typischen Desktoprechnern tausende von E/A-Operation durchführen und auf den heutigen HPC-System können es sogar hunderttausende oder Millionen sein. 
Die zu analysierende Datenmengen werden dabei nicht selten hunderte von Gigabytes gross. 
Um das komplexe Zusammenspiel der E/A-Zugriffe dennoch zu verstehen und um die Ursache dafür zu identifizieren werden bringt man den Tools immer cleverere Analyse-Techniken bei.

\subsection{Darshan}
Darshan ist ein HPC E/A-Charakterisierungstool.
Es wurde entworfen um ein genaues Bild vom Anwendungsverhalten und Eigenschaften, wie Zugriffsmuster in einer Datei, mit minimum Overhead aufzunehmen.

Darshan kann benutzt werden um das E/A-Verhalten von HPC-Anwendungen zu untersuchen und zu optimieren. 
Sein leichter Design macht ihn auch optimal für die durchgehende Lastcharakterisierung in großen Systemen während des produktiven Betriebs.

\subsection{Vampir}
%%% Vampir %%% 
% Kurze Beschreibung von Vampir
% graphische Software
% unterstuetzt MPI, OpenMP, multithreaded Applications
Vampir \todo{Referenz} ist der Marktführer in E/A-Leistungsanalyse von parallelen Systemen. 
Es unterstützt die Offline-Analyse von parallelen (MPI, OpenMP, Multithreading) und hardwarebeschleunigten (CUDA und OpenCL) Anwendungen. 
Seine Analyseengine erlaubt eine skalierbare und effiziente Verarbeitung von sehr großen Mengen von Leistungsdaten und macht ihn besonders geeignet für den Einsatz auf HPC-Systemen.

% Funktionsweise
% erstellt Traces
Vampir nutzt die Infrastruktur von Score-P \todo{Referenz} zum Instrumentieren von Anwendungen. 
Score-P speichert die Aktivitäten in einer Datei ab, die von Vampir analysiert und in verschieden Sichten umgewandelt werden kann, z.B. können die Aktivitäten auf einer Zeitachse dargestellt werden oder zu einer der vielen Statistiken komprimiert werden. 
Einige Sichten verfügen über aufwendige Filter- und Zoom-Funktionen, mit den man sich leicht einen Überblick verschaffen und die Details betrachten kann.  

% Vor- und Nachteile
Eine effektive Nutzung von Vampir erfordert ein tiefes Hintergrundwissen in der parallelen Programmierung. 
Das Framework bietet zwar sehr aussagekräftige und qualitative Informationen über das Geschehen auf der E/A-Ebene, aber wenig Informationen über deren Zustandekommen. 
Der Einsatzbereich von Vampir wird durch die fehlende Unterstützung von Online-Analyse etwas eingeschränkt.  

% Zweck:
% Senken von Kommunikationskosten
% Balancierung von Kommunikation zwischen CPU und Cache oder Arbeitsspeicher
% Engpaesserkennug auf Kommunikationsstrecken
%Vampir eignet sich wenn es darum geht die Kommunikationskosten zu senken, die Kommunikation wischen CPU und Cache bzw. Arbeitsspeicher auszubalancieren oder Engpässe auf Kommunikationsstrecken zu erkennen.


\subsection{SIOX-Framework}

% Allgemeine Beschreibung
% Unterstützung weiterer Interfaces
% Activitaetgenerator
% Monitoring Plugins
SIOX \cite{siox_arch} ist ein unter GPL veröffentlichte Open-Source-Framework, der seit dem Jahr 2011 im Rahmen eines Projektes an der Universität Hamburg entwickelt wird. 
Es unterstützt die ereignisbasierte Analyse von E/A-Zugriffe auf Desktop- und HPC-Systemen. 
Das Framework enthält ein Instrumentierungswerkzeug, ein Trace-Reader, diverse Analyse-Plugins und die Generatoren von Modulen zur Instrumentierung vom MPI- und Posix-Schnittstellen. 

% Funktionsweise
% Aktivitätene und Sequenzen
Ähnlich wie Vampir, arbeitet SIOX mit Aktivitäten und Aktivitätssequenzen, die beim Ausführen der instrumentierten Anwendungen oder künstlich mit einem Plugin erzeugt werden können. 
Die Aktivitäten sind Datenstrukturen, die relevanten Daten und Messungen über die E/A-Operation beinhalten. 
Sie können zur späteren Analyse in einer Datei gespeichert oder in Echtzeit verarbeitet werden.

% Plugins
Im Kern stellt SIOX eine Plugin-Infrastruktur bereit und erhält seine entgültige Funktionalität erst durch die Anbidung von Plugins und Modulen. 
Das macht ihn leicht erweiterbar und fast unbegrenzt an die eigene Bedürfnisse anpassbar. 
So kann man z.B. eigene Module zur Instrumentierung von Schnittstellen anbinden oder die Aktivitäten künstlich mit einem eigenen Aktivitätsgenerator-Plugin erzeugen und die erzeugten Aktivitäten mit eigenen Analyse-Plugins verarbeiten.

% GUI und Ausgabe
In SIOX ist es möglich Statistiken im Web-Interface anzuzeigen. 
Diese Funktion ist momentan nur auf die Auflistung von Aktivitäten beschränkt, kann aber leicht erweitert werden. 
Komplexere Ausgaben und Auswertungen können durch die Plugins vorgenommen werden.

% Vorteile und Nachteile
SIOX unterstützt sowohl Online- als auch Offline-Analyse. 
Zusätzlich erlaubt die Anbindung eines Aktivitätsgenerator-Plugins auch eine hypothetische Situation zu simulieren. 
Es gibt allerdings noch sehr wenige Plugins und die Wahrscheinlichkeit darunter einen passenden zu finden ist sehr gering. 
Das Framework befindet sich im Entwicklungsstadium und hat keine feste Schnittstelle.


\subsection{Kombination vom analytischen und maschinellen Lernen}
In \cite{Didona:2015:EPP:2668930.2688047} versuchen die Autoren eine Verbesserung der Leistungsvorhersage mit einer Kombination aus Machine-Learning und Analytical Modeling zu erreichen und Vorteile aus beiden Modellen zu nutzen. 
Aus dem analytischen Modell kann leicht herauslesen, wie die Entscheidung zustande kommt, aber das Modell liefert keine so gute Vorhersagen wie Machine-Learning. 
Das Machine-Learning liefert zwar wesentlich bessere Ergebnisse, ist aber das Zustandekommen der Entscheidung ist sehr schwer nachvollziehbar. \todo{Zusammenfassen}
