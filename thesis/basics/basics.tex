
%2. 31.05-05.06
%Related Work und Hintergrund (7 Seiten)
%* Machine Learning (Features, Labels), Classification, Clustering,
%K-Cross-Validation
%* Verhalten von I/O Systemen (Caches, Wie läuft I/O Kommunikation ab)
%* SIOX => Architektur aus Papern bzw. dein Wissen klauen.
%Bisher kein Vollständiger Wissenszyklus (Messen => Lernen =>
%Optimieren => Optimieren ...)
\section{Verwandte Arbeiten und Hintergrund}
\todo[inline]{
2. 31.05-05.06\\
Related Work und Hintergrund (7 Seiten)\\
* Machine Learning (Features, Labels), Classification, Clustering,\\
K-Cross-Validation\\
* Verhalten von I/O Systemen (Caches, Wie läuft I/O Kommunikation ab)\\
* SIOX => Architektur aus Papern bzw. dein Wissen klauen.\\
Bisher kein Vollständiger Wissenszyklus (Messen => Lernen =>\\
Optimieren $=>$ Optimieren ...)
}
\todo[inline]{6 Paper lesen, 6 Arbeiten vorstellen}
\todo[inline]{Vielleicht: IO Performance Prediction in Consolidated Virtualized Environments}
\todo[inline]{Auf jeden Fall: Enhancing Performance Prediction Robustness by Combining Analytical Modeling and Machine Learning}
In \cite{Didona:2015:EPP:2668930.2688047} versuchen die Authoren eine Verbesserung der Leistungsvorhersage mit einer Kombination aus Machine-Learning und Analytical Modeling zu erreichen und Vorteile aus beiden Modellen zu nutzen. Aus dem analytischen Modell kann leicht herauslesen, wie die Entscheidung zustande kommt, aber das Modell liefert keine so gute Vorhersagen wie Machine-Learning. Das Machine-Learning liefert zwar wesentlich bessere Ergebnisse, ist aber das Zustandekommen der Entscheidung ist sehr schwer nachvollziehbar. 
	
\subsection{Machine-Learning}
Dieser Abschnitt stellt einige Begriffe aus dem Machine-Learning-Bereich vor. Die Definitionen sind and das Buch von Peter Flach \cite{Flach:2012:MLA:2490546} angelehnt.

Ein \textbf{Merkmal} bzw. ein \textbf{Feature} $f_i$ beschreibt eine Eigenschaft des untersuchten Gegenstanden, z.B. kann es das Alter einer Person, die Hausnummer oder die Datengröße sein. In dieser Arbeit wird durchgängig der Begriff Feature verwendet. Mathematisch gesehen handelt es sich um eine Abbildung von dem Instanzraum auf die Featuredomain $f_i : \mathscr{X} \rightarrow \mathscr{F}_i$. Mehrere Features bilden ein Featurevektor $F$. 

Die \textbf{Featuredomain} legt alle Werte fest, die ein Feature annehmen kann. In dieser Arbeit werden wir stets von numerischen Werten ausgehen.

Der \textbf{Instanzraum} ist ein Kreuzprodukt aus den betrachteten Featuredomains $\mathscr{X} = \mathscr{F}_1 \times \dots \times \mathscr{F}_d$, wobei die Anzahl der Features $d$ die Dimension des Instanzraumes festlegt.

TODO: Regressions- und Multiclasslabels

TODO: Ein Sample $S = (F, L)$ ist ein Tupel aus einem Featurevektor und einem Label. Mehrere Samples können zu einem Sampleset $\mathcal{S}$ zusammengefasst werden.

TODO: Classification
TODO: Clustering

k-Fold Cross-Validation ist eine Auswertungsmethode der Machine-Learning-Algorithmen. Eine Datenmenge wird in $k$ gleiche grosse Mengen aufgeteilt. Eine Menge wird für das Training benutzt und $k-1$ Mengen zum Vorhersagen der Werte. Die Differenze zwischen den echten und vorhergesagten Werten werden zu einem Durchschnittsfehler zusammengefasst. Dieser Schritt wird für die restlichen Mengen durchgeführt und von den $k$ berechnenten Durchschnittsfehlern wird ein Durchschnitt berechnet. Dieser Wert beschreibt schließlich wie gut der Algorithmus das Problem lösen kann.

\subsection{E/A-Systeme}


\subsection{Hardware}
Busse
Festplatten- und SSD-Technik
Netzwerk


\subsection{Caches}
Zweck

Tradidionelle Formen:
Register
Prozessor Cache
Arbeitsspeicher
Festplattencache

Sonderformen: Festplatte als Cache, SSD-Cachesysteme, Auslagerungsdatei, HTTP-Caching


\subsection{Leistungskennzahlen}
Wegen den unterschiedlichen Einsatzzwecken lässt sich die Leistung der E/A-Systeme nicht durch eine einzige Kennzahl bemessen. Stattdessen ist die Leistungskennzahl abhängig vom Systemtypen und muss für jedes System individuell bestimmt werden. Die gängigsten Leistungskennzahlen sind:
\begin{enumerate}
	\item Datendurchsatz
	\item Antwortzeit
	\item Auslastung
\end{enumerate}


\subsection{Einflussgrößen}
Beeinflusst wird die Leistung von E/A-Systemen durch ein komplexes Zusammenspiel von verschiedenen Faktoren.

\begin{enumerate}
	\item Blockgrößen
	\item Parallele Zugriffe auf Disk-Subsysteme
	\item Betriebssystem und Anwendungen
	\item Controller
	\item Speichermedium
\end{enumerate}

% Datenbloecke
Die Daten werden immer blockweise von Massenspeichermedium gelesen. Die Blockgröße ist abhängig von Systemkonfiguration, der Anwendung oder Betriebssystem. Bei einem eingerichteten System ist die Blockgröße vorgegeben und kann der Benutzer in den meisten Fällen nicht bestimmt werden. 

% Parallele Zugriffe
Für parallele Zugriffe haben die meisten System einen Wartenschlangenmechanismus, der die Aufträge zwischenspeichert. Verschiedene Kontroller bzw. Konfiguration können aber verschiedene Strategien verfolgen. (1) Reduktion der Antwortzeiten mit sequentiellen Abarbeitung der Aufträgen und (2) Erhöhung des Durchsatzes mit parallelen Ausführung von einer bestimmter Anzahl von Aufträgen.

Parallele Zugriffe auf ein Massenspeichermedium führen immer zu steigenden Verzögerungszeiten.


% Betriebssystem und Anwendungen

\subsection{SIOX}

SIOX is a framework for analysis of IO-Performance of computers. The SIOX project started in 2011 and is actively developed at the University of Hamburg. 

The extreme modular architecture of SIOX makes to a versatile and extensible framework. It provides several interfaces for plugins and gives the user a possibility to adapt SIOX to his own needs.
