%06.06-13.06
%3. Design => ca. 15 Seiten
%- Optimierungs/Wissenszyklus Abstrakt
%- Leistungs-Vorhersage (Predictor)
%- Reduktion von Werten zu Klassen => Preprocessing
%- Ursachen-Klassifikator (Cached vs. Uncached mit Labels die von
%Benutzern zu definieren sind). Beispielgraph einfach von Daten
%übernehmen.
%- Extraction von Wissen aus Decision Trees => verweis nach hinten.
%- Einsatz von Decision Tree (parameterisierung) ??? Sonst Implementierung.
%- Beispielhafte Ausgabe eines Programlaufs
%=> Bei Terminierung des Programs wollen wir ja sehen, dass der die
%Gründe und Bewertung vornimmt. Wieviel I/O In Klasse 1, 2....
%(Entsprechend der Namen).
\section{Design}
Das Design des Programms soll darauf ausgerichtet werden, um uns den 
\textit{Optimierung/Wissenszyklus, Leistungsprädiktor, Ursachen-Klassifikator, Wissensextraktion, Ausgabe des Plugins}


\section{SIOX-Aktivitäten}
Die SIOX-Aktivitäten sind Datenstrukturen, die Informationen über die E/A-Operationen beinhalten.
Die aktuelle Version arbeitet bereits mit HDF5-, NETCDF4-, MPI- und POSIX-Schnittstellen.
Die Aktivitäten dienen als eine Basis für die Leistungsanalyse und werden in diesem Abschitt detailiert betrachtet.

Die \tabref{tab:des:activity} listet grob alle Bestandteile einer SIOX-Aktivität.
Der Typ und der Name wurden direkt aus dem Quellcode übernommen und kurz beschrieben.
Da in dieser Arbeit nur die POSIX-Schnittstelle betrachtet wird, werden nicht alle Bestandteile benötigt.
Die nicht verwendeten Bestandteile wurde in der Tabelle ausgegraut.

\begin{table}[h]
	\centering
	\begin{scriptsize}
		\begin{tabular}{l | l | l}
			Typ & Name & Beschreibung \\
			\hline
			\lstinline{ActivityID}                                              & \lstinline{aid}                                                & Eindeutiger Bezeichner \\
			\lstinline{UniqueComponentActivityID}                               & \lstinline{ucaid}                                              & Typ der E/A-Operation \\
			\lstinline{Timestamp}                                               & \lstinline{time_start}                                         & Startzeit in Nanosekunden\\
			\lstinline{Timestamp}                                               & \lstinline{time_stop}                                          & Stopzeit in Nanosekunden \\
			\lstinline{vector<ActivityID>}                                      & \lstinline{parentArray}                                        & Ausgangs-E/A-Operationen \\
			\lstinline[basicstyle=\ttfamily\color{gray}]{vector<RemoteCall>}    & \lstinline[basicstyle=\ttfamily\color{gray}]{remoteCallsArray} & \textcolor{gray}{\textit{-}} \\
			\lstinline{vector<Attribute>}                                       & \lstinline{attributeArray}                                     & Parameter, Rückgabewert, usw. \\
			\lstinline[basicstyle=\ttfamily\color{gray}]{RemoteCallIdentifier*} & \lstinline[basicstyle=\ttfamily\color{gray}]{remoteInvoker}    & \textcolor{gray}{\textit{-}} \\
			\lstinline[basicstyle=\ttfamily\color{gray}]{ActivityError}         & \lstinline[basicstyle=\ttfamily\color{gray}]{errorValue}       & \textcolor{gray}{\textit{-}}
		\end{tabular}
	\end{scriptsize}
	\label{tab:des:activity}
	\caption{Zusammensetzung der SIOX-Aktivitäten. (Die ausgegrauten Bestandteile sind nicht relevant für diese Arbeit.)}
\end{table}

Ganz besonders interessant sind die drei POSIX-Operationstypen: Eingabe-, Ausgabe- und Positionierungsoperationen, denn in den meisten Fällen bestimmen sie die E/A-Leistung.
%https://en.wikipedia.org/wiki/C_file_input/output
Um einmal zu zeigen wie diese Operationen in den SIOX-Aktivitäten behandelt werden schauen wir uns jeweils einen Vertreter von jedem Typ an, nämlich \lstinline{fread}-, \lstinline{fwrite}- und \lstinline{fseek}-Operation (\lstref{lst:des:posixops}).

\begin{lstlisting}[language=C, belowcaptionskip=1\baselineskip, caption={POSIX-Operationen}\label{lst:des:posixops}]
size_t fread(void *ptr, size_t size, size_t nmemb, FILE *stream);
size_t fwrite(const void *ptr, size_t size, size_t nmemb, FILE *stream);
int fseek(FILE *stream, long offset, int whence);
\end{lstlisting}

Die \lstinline{aid}, \lstinline{time_start} und \lstinline{time_stop} Elemente sind von Funktion her trivial. 
ucaid ist als eine Zahl kodierter Operationstyp und ist ebenfalls trivial. 
Spannend wird es bei \lstinline{parentArray} und \lstinline{attributeArray}.

Das Attribute \lstinline{parentArray} verweist auf die Ausgangs-Aktivitäten, d. h. auf die Aktivitäten mit der der Zugriff auf eine bestimmte Datei begonnen hat.
Bei POSIX handelt es sich typischerweise um die open-Aktivität.
Die Aktivitäten mit dem gleichen parentArray bilden eine Gruppe und, wenn zeitlich angeordnet, eine Aktivitätensequenz auf eine bestimmte Datei.
Wenn eine Datei mehrmals geöffnet wurde und die Zugriffe über verschiedene FileHandler erfolgen, dann sind auch die parentArray's unterschiedlich.
Hat eine Aktivität keine Ausgangsaktivität, wie z. B. open-Aktivität, dann ist parentArray leer.


Das Attribute attributeArray beinhaltet Informationen über den Systemaufruf.
Das können die Funktionsparameter, der Rückgabewert, diverse Laufzeitinformationen usw. sein.
Der Inhalt von attributeArray ist spezifisch für die E/A-Operation. 

\begin{table}[h]
	\centering
	\begin{tabular}{l | l}
		Attribute & Wert\\
		\hline
		POSIX/quantity/BytesToWrite & 4\\
		POSIX/descriptor/FilePointer & 28317504\\
		POSIX/data/MemoryAddress & 4233230\\
		POSIX/quantity/BytesWritten & 4
	\end{tabular}
	\caption{Attribute und Beispielwerte einer write-Operation.}
\end{table}

\begin{table}[h]
	\centering
	\begin{tabular}{l | l}
		Attribute & Wert\\
		\hline
		POSIX/quantity/BytesToRead & 100\\
		POSIX/data/MemoryAddress & 22192128\\
		POSIX/descriptor/filehandle & 0\\
		POSIX/quantity/BytesRead & 100\\
	\end{tabular}
	\caption{Attribute und Beispielwerte einer read-Operation.}
\end{table}

\begin{table}[h]
	\centering
	\begin{tabular}{l | l}
		Attribute & Wert\\
		\hline
		POSIX/descriptor/filehandle & 4\\
		POSIX/file/position & 0\\
	\end{tabular}
	\caption{Attribute und Beispielwerte einer read-Operation.}
\end{table}



\section{Erzeugung von Aktivitätensequenzen}

Die Offline-Analyse wird in drei Schritten durchgeführt.

Im ersten Schritt werden die Aktivitäten einer Anwendung aufgezeichnet und in einer Spurdatei gespeichert \autoref{fig:des:get_trace}. 
Dazu wir das Siox-Instrumentierungstool verwendet.
\begin{figure}[h]
	\centering
	\begin{tikzpicture}
	[
		remember picture,
		>=latex reversed,
		organism/.style={draw,rounded corners},
		node distance=5mm and 5mm
	]
	
	\node
		[organism]
		(app)
		{Anwendung};
	\node
		[organism,right=of app]
		(inst)
		{Instrumentierung};
	\node
		[organism,right=of inst]
		(tracefile)
		{Spurdatei};

	\path
		[->, >=latex]
		(app)	edge	(inst)
		(inst)	edge	(tracefile);


\end{tikzpicture}

	\label{fig:des:get_trace}
	\caption{Erstellung einer Spurdatei.}
\end{figure}

Im zweiten Schritt werden die Aktivitäten ausgewertet. 
Dazu wird die erstellt Spurdatei mit siox-trace-reader eingelesen, der dann die Aktivitäten an den Analyse-Plugin weiterleitet.
Das Plugin startet dann den Lernprozess.
Die Ergenisse im diesen Schritt sind zwei Entscheidungsbäume: das eine für die Leistungsvorhersage und das andere für die Klassifizierung.

\begin{figure}[h]
	\centering
	\begin{tikzpicture}
	[
		remember picture,
		>=latex reversed,
		organism/.style={draw,rounded corners},
		node distance=5mm and 5mm
	]
	
	\node
		[organism]
		(file)
		{Spurdatei};
	\node
		[organism,right=of file]
		(tracereader)
		{SIOX};
	\node
		[organism,right=of tracereader]
		(ke)
		{Analyse-Plugin};
	\node
		[organism,right=of ke]
		(result)
		{Ergebnisse};

	\path
		[->, >=latex]
		(file)	edge	(tracereader)
		(tracereader)	edge	(ke)
		(ke) edge (result);


\end{tikzpicture}

	\label{fig:des:results}
	\caption{Vorhersage}
\end{figure}

Im dritten Schritt wird wieder eine Spurdatei (die gleiche oder eine andere) eingelesen und vom Plugin ausgewertet.
Die Ergebnisse sind Vorhersagen der Leistungswerte, Klassenzuordnung und Anomalien.

Die Anomalien liefern Hinweise auf besondere Stellen.
Diese Stellen werden untersucht und nach Möglichkeit Rückschlüsse gezogen.

\subsection{Optimierungs/Wissenszyklus Abstrakt}
Die in der Arbeit verwendete Lehrnverfahren unterstützen keine Fortsetzung der Trainingsphase.
D.h. es müssen erst die Trainingsdaten gesammelt werden und dann kann das Modell trainiert werden.
Fertige Modelle können nachträglich mit weiteren Daten nicht verbessert werden.
Diese Eigenschaft der Lernverfahren spiegelt sich direkt in der Funktionsweise des Plugins. 
In der ersten Phase \figref{fig:des:perf_training} erzeugt das Plugin aus der Aktivitätensequenz Trainingsdaten, trainiert damit die Modelle und speichert sie in einer Datei ab.
In der zweiten Phase erzeugt das Plugin aus der Aktivitätensequenz eine Testmenge und wendet sie an.

\begin{figure}[h]
	\hfill
	\subfigure[Training]{
		\input{pictures/perf_training}
		\label{fig:des:perf_training}
	}
	\hfill
	\subfigure[Vorhersage]{
		\begin{tikzpicture}
	[
		%remember picture,
		>=latex reversed,
		node distance=5mm and 5mm,
		flowchart
	]

	\node
		[block]
		(start)
		{Start};
	\node
		[block, below=of start]
		(load)
		{lade Modell};
	\node
		[block,below=of load]
		(get)
		{nehme Aktivität};
	\node
		[block,below=of get]
		(compute)
		{erzeuge Beispiel};
	\node
		[block,below=of compute]
		(predict)
		{schätze Leistung};
	\node
		[block,below=of predict]
		(anomaly)
		{prüfe auf Anomalie};
	\node
		[decision, below=of anomaly]
		(decide)
		{Sequenzende?};
	\node
		[block, below=of decide]
		(results)
		{gebe Statistiken aus};
	\node
		[block, below=of results]
		(end)
		{Ende};

	\path
		[->, >=latex]
		(start) edge (load)
		(load)	edge	(get)
		(get) edge (compute)
		(compute) edge (predict)
		(predict) edge (anomaly) 
		(anomaly) edge (decide)
		(decide) edge node[near start, right] {ja} (results)
		(results) edge (end);
	
	\path
		[line]
		(decide) edge [out=180, in=180] node[near start, left] {nein}  (get);

\end{tikzpicture}


		\label{fig:des:perf_prediction}
	}
	\hfill
	\label{fig:des:perf_phases}
	\caption{Arbeitsweise vom Leistungsprädiktor. In der Trainingsphase werden Daten gesammelt und das Modell trainiert. In der Analysephase wird das Modell zur Leistungsvorhersage, Anomalieerkennung und Statistikerzeugung verwendet.}
\end{figure}

\subsection{Leistungsprädiktor}
Die grundlegende Aufgabe des Leistungsprädiktors die Informationen über den Systemzustand zu nutzen, um die Leistung vorherzusagen.
Damit der Leistungsprädiktor seine Aufgabe erfüllen kann, wird eine Maschinenlernalgorithm benutzt, um ein Regressionsmodell zu trainieren.
Das Modell wird dann in der Lage sein, die bereits bekannte Werte und neue Werte vorherzusagen.

Das Problem kann auch in ein Klassifikationsproblem umgewandelt werden. 
Der Wertebereich kann in Intervalle geteilt werden und jedes Intervall bekommt seine eigene Klasse.
Der Nachteil bei dieser Vorhegensweise ist, dass keine neuen Werte vorhergesagt werden können.
Der Vorteil wäre, man könnte auch beliebige Klassifikationsalgorithmen nutzen und den Lernvorgang beeinflussen.





\subsection{Ursachen-Klassifikator}
Die Aufgabe des Ursacheklassifikators ist ähnliche E/A-Operation zu gruppieren und die Gruppen zu erlernen, um später die E/A-Operation klassifizieren zu können.

In ersten Phase wird der Trainingsset mit einem Clusteralgorithmus in Gruppen eingeteilt und die erzeugten Gruppen werden den entsprechenden Vektoren zugeordnet.
Daraus entsteht dann ein neues Trainingsset, der zweiten Phase vom Klassifizierungsalgorithmus erlernt wird.

Die Gruppen müssen richtig bennant werden.

\begin{figure}[h]
	\hfill
	\subfigure[Training]{
		\begin{tikzpicture}
	[
		%remember picture,
		>=latex reversed,
		node distance=5mm and 5mm,
		flowchart
	]

	\node
		[block]
		(start)
		{Start};
	\node
		[block, below=of start]
		(get)
		{nehme Aktivität};
	\node
		[block,below=of get]
		(compute)
		{erzeuge Beispiel};
	\node
		[block,below=of compute]
		(save)
		{füge zu Trainingsdaten hinzu};
	\node
		[decision, below=of save]
		(decide)
		{Sequenzende?};
	\node
		[block, below=of decide]
		(cluster)
		{gruppiere Trainingsdaten};
	\node
		[block, below=of cluster]
		(traindata)
		{füge Gruppen\\zu Trainingsdaten hinzu};
	\node
		[block,below=of traindata]
		(train)
		{trainiere Modell};
	\node
		[block, below=of train]
		(results)
		{speichere Modell};
	\node
		[block, below=of results]
		(end)
		{Ende};

	\path
		[->, >=latex]
		(start) edge (get)
		(get) edge (compute)
		(compute) edge (save)
		(save) edge (decide)
		(decide) edge  node[near start, right] {ja} (cluster)
		(cluster) edge (traindata)
		(traindata) edge (train) 
		(train) edge (results)
		(results) edge (end);
	
	\path
		[line]
		(decide) edge [out=180, in=180] node[near start, left] {nein}  (get);

\end{tikzpicture}


		\label{fig:des:class_training}
	}
	\hfill
	\subfigure[Vorhersage]{
		\begin{tikzpicture}
	[
		%remember picture,
		>=latex reversed,
		node distance=5mm and 5mm,
		flowchart
	]

	\node
		[block]
		(start)
		{Start};
	\node
		[block, below=of start]
		(load)
		{lade Modell};
	\node
		[block,below=of load]
		(get)
		{nehme Aktivität};
	\node
		[block,below=of get]
		(compute)
		{erzeuge Beispiel};
	\node
		[block,below=of compute]
		(predict)
		{schätze Gruppe};
	\node
		[decision, below=of predict]
		(decide)
		{Sequenzende?};
	\node
		[block, below=of decide]
		(results)
		{gebe Statistiken aus};
	\node
		[block, below=of results]
		(end)
		{Ende};

	\path
		[->, >=latex]
		(start) edge (load)
		(load)	edge	(get)
		(get) edge (compute)
		(compute) edge (predict)
		(predict) edge (anomaly) 
		(anomaly) edge (decide)
		(decide) edge node[near start, right] {ja} (results)
		(results) edge (end);
	
	\path
		[line]
		(decide) edge [out=180, in=180] node[near start, left] {nein}  (get);

\end{tikzpicture}


		\label{fig:des:class_prediction}
	}
	\hfill
	\label{fig:des:class_phases}
	\caption{Arbeitsweise vom Ursachen-Klassifikator. In der ersten Trainingsphase werden Daten gesammelt und das Modell trainiert. In der Analysephase wird das Modell für die Gruppenzuordnung und die Statistikerzeugung verwendet.}
\end{figure}

\subsection{Ausgabe}
Beispielausgabe\\
k-Cross-Validation\\
E/A-Operationstypen\\
Benennung der Gruppen (cached, uncached)\\
Gruppen und Durchsatz

\subsection{Anomalieerkennung}

\subsection{Extraktion von Wissen}
%Anomalien
Eine Anomalie tritt auf, wenn der vorhergesagter Wert von dem gemessenen Wert sich erheblich unterscheidet.
Es können sowohl gute Anomalien auftretten, wo der gemesserner Wert wesentlich besser ist als der vorhergesagter und schlechte Anomalien im Umgekehrten Fall.
Anhand dieser Anomalien kann man erkennen welche Situationen man vermeiden soll und welche anstreben.

%Entscheidungsbaeume
Kleine Entscheidungsbäume kann man manuell durchgehen und nachvollziehen wie die Entscheidung zustande gekommen ist.

\textit{Zusammenfassung}
